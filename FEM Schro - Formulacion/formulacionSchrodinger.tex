\documentclass[12pt,letterpaper,final]{article}
\usepackage[english]{babel}
\usepackage[ansinew]{inputenc}
\usepackage{amsmath}
\usepackage{amsfonts}
\usepackage{amssymb}
\usepackage{url}
\usepackage[top=3cm, bottom=2.5cm, left=2.5cm, right=2.5cm]{geometry}
\hyphenation{tem-pe-ra-tu-ra}
\usepackage[pdftex,pdftitle={Formulaci\'on FEM para ecuaci\'on de Schr\"{o}dinger},pdfauthor={Monitores y el Nicolas},
  pdfsubject={Deducci\'on FEM para Ec. Schr\"{o}dinger},pdfkeywords={Seminario, Proyecto, Semillero},
  pdfpagemode=UseOutlines,bookmarks,bookmarksopen,pdfstartview=FitH,colorlinks,linkcolor=black,citecolor=black,
  urlcolor=black,]{hyperref}
\setlength{\parskip}{0.0mm}
\parindent=0mm

\begin{document}
\title{\textbf{FEM formulation for Schr\"{o}dinger equation}}
\author{ Nicol\'as Guar\'in
}

\date{\today}
\maketitle

\section{Weak form deduction}
Consider the time-dependent Schr\"{o}dinger equation
\[i \hslash \frac{\partial}{\partial t}\Psi = \hat{H}\Psi ,\]
being $\hat{H}$ the Hamiltonian Operator ($-\frac{\hslash^2}{2m}\nabla^2 + V(\vec{x})$). $-\frac{\hslash^2}{2m}\nabla^2$ is the kinetic energy operator and $V(\vec{x})$ is the potential energy. \\

If we assume 
\[\Psi = \Psi e^{-i\omega t}\]
we get 
\[i \hslash (-i \omega e^{-i \omega t}\Psi)= \hat{H} e^{-i\omega t}\]
which yields 
\[\hslash \omega \Psi = \hat{H} \Psi\]

and is the time-independent Shcr\"{o}dinger equation, usually written as
\begin{equation}
E \Psi = \hat{H}\Psi \quad (E=\hslash \omega = h \nu) \enspace .
\end{equation}

For the FEM solution, let's multiply by a test function $v$ and integrate over the domain $\Omega$

\[\int\limits_{\Omega}E \Psi v d\Omega = \int\limits_{\Omega}\hat{H}\Psi v d\Omega\]
or
\begin{equation}
E \int_{\Omega}\Psi (x)v(x)d\Omega = -\frac{\hslash^2}{2m}\int_{\Omega}\nabla^2\Psi(x)v(x)d\Omega + \int_{\Omega}V(x)\Psi(x)v(x)d\Omega \enspace .
\label{fp}
\end{equation}

\eqref{fp} could be rewritten, via Green's theorem, as
\[E \int\limits_{\Omega} \Psi(x) v(x) d\Omega  =
\frac{\hslash^2}{2m} \int\limits_{\Omega}\nabla \Psi(x)\nabla v(x) d\Omega + \int\limits_{\Omega}V(x)\Psi(x)v(x)d\Omega - 
\int\limits_{\Gamma}v(x)\frac{\partial \Psi (x)}{d \vec{n}} d\Gamma \quad \mbox{with } \Gamma=\partial\Omega\]
The term $\int_{\Omega}v(x)\frac{\partial \Psi (x)}{d \vec{n}}\partial \Omega$ won't make any contribution to the discrete operators (matrices) of the system, so let us consider just

\begin{equation}
E \int\limits_{\Omega}\Psi (x)v(x)d\Omega  = \frac{\hslash^2}{2m} \int\limits_{\Omega}\nabla \Psi(x)\nabla v(x) d\Omega + \int\limits_{\Omega}V(x)\Psi(x)v(x)d\Omega
\label{fd}
\end{equation} 
 
\section{Approximated functions: interpolation}
Functions $\Psi (x)$ and $v(x)$ need to be approximated as

$$\Psi (x) = \sum_{i=1}^{n}\Psi_iN_i(x), \quad \text{with } \Psi_i=\Psi(x_i);$$
$$v (x) = \sum_{i=1}^{n}v_i N_i(x), \quad \text{with } v_i=v(x_i);$$
being these interpolation calculated over the (disjoint) elements.

Since $V(x)$ is the potential, it is known, and two approximations could be applied:

\begin{itemize}
\item Assume $V(x)$ constant over each element, or
\item Interpolate it with the form functions $N_i(x)$.
\end{itemize}

Equation \eqref{fd} could be stated as

\[\frac{\hslash^2}{2m} \sum_{el=1}^{N_{el}} \int\limits_{\Omega}\nabla \Psi(x)\nabla v(x) d\Omega + \sum_{el=1}^{N_{el}} \int\limits_{\Omega}V(x)\Psi(x)v(x)d\Omega  =  E \sum_{el=1}^{N_{el}} \int\limits_{\Omega}\Psi (x)v(x)d\Omega  \enspace ,  \]
where $el$ is a label for the elements.

Therefore, we need to compute it for a single element and then assemble.

\[\frac{\hslash^2}{2m} \sum_{el=1}^{N_{el}} \int\limits_{\Omega}\nabla \Psi(x)\nabla v(x) d\Omega + \sum_{el=1}^{N_{el}} \int\limits_{\Omega}V(x)\Psi(x)v(x)d\Omega  =  E \sum_{el=1}^{N_{el}} \int\limits_{\Omega}\Psi (x)v(x)d\Omega  \enspace ,  \]

So, for each element we have
\begin{equation}
\frac{\hslash}{2m}\int\limits_{\Omega_{el}}\nabla \Psi(x) \nabla v(x) d\Omega_{el} +
\int\limits_{\Omega_{el}} V(x)\Psi(x) v(x) d\Omega_{el} =
E\int\limits_{\Omega_{el}} \Psi(x) v(x)d\Omega_{el} \enspace .
\label{eq:weak_form-element}
\end{equation}

Then, the approximate functions are replaced in \eqref{eq:weak_form-element}. Two approximated
versions of equation \eqref{eq:weak_form-element} are given
\begin{itemize}
\item assuming the potential $V(x)$ constant over the element, or
\item using the same interpolation functions for $V(x)$.
\end{itemize}

\subsection{Potential constant over the element}
The resulting equation is
\begin{equation}
\frac{\hslash}{2m}\int\limits_{\Omega_{el}} \Psi_i v_j \nabla N_i(x) \nabla N_j(x) d\Omega_{el} +
V(x_m)\int\limits_{\Omega_{el}} \Psi_i v_j N_i(x) N_j(x) d\Omega_{el} =
E\int\limits_{\Omega_{el}} \Psi_i v_j N_i(x) N_j(x)d\Omega_{el} \enspace ,
\label{eq:weak_form-constant}
\end{equation}
where $x_m$ is the centroid of the element.

\subsection{Potential interpolated over the element}
The resulting equation is
\begin{equation}
\frac{\hslash}{2m}\int\limits_{\Omega_{el}} \Psi_i v_j \nabla N_i(x) \nabla N_j(x) d\Omega_{el} +
\int\limits_{\Omega_{el}} V_i \Psi_i v_j N_i^2(x) N_j(x) d\Omega_{el} =
E\int\limits_{\Omega_{el}} \Psi_i v_j N_i(x) N_j(x)d\Omega_{el} \enspace .
\label{eq:weak_form-varriable}
\end{equation}

\section{Discrete form}
Equation \eqref{eq:weak_form-element} could be written in an abstract (inner-product) fashion
like
\[ \langle\mathbb{K}\bold{\Psi},\bold{v}\rangle + \langle\mathbb{V}\bold{\Psi},\bold{v}\rangle
=E\langle\mathbb{M}\bold{\Psi},\bold{v}\rangle \enspace ,  \]
this is the same that
\[ \langle \mathbb{K}\bold{\Psi} + \mathbb{V}\bold{\Psi} - E\mathbb{M}\bold{\Psi} ,
\bold{v} \rangle = 0 \enspace ,\]
and since $v(x)$ is an \emph{arbitrary} function the vector $\bold{v}$ it's too. As a
consequence we have
\[\mathbb{K}\bold{\Psi} + \mathbb{V}\bold{\Psi} = E\mathbb{M}\bold{\Psi} ,\]

The final discrete system, with boundary conditions, will take the form

\[\mathbb{K}\bold{\Psi} + \mathbb{V}\bold{\Psi} = E \mathbb{M}\bold{\Psi} + \bold{d}+ \bold{q}\]

where $\mathbb{K}$ is the equivalent \emph{kinetic energy operator} (stiffness matrix), $\mathbb{V}$ is the \emph{potential energy operator}, $\mathbb{M}$ is the \emph{momentum operator} (mass matrix), $\bold{d}$ is the vector with Dirichlet's conditions and $\bold{q}$ is the vector with Neumann's conditions.

In quantum mechanics Neumann's boundary conditions don't have a great physical meaning and the
most common Dirichlet's conditions are the homogeneous ones, furthermore one is interested in the values of $E$ and the functions $\Psi$ that satisfy the original equation. So the discrete
system to solve is
\begin{equation}
\mathbb{K}\bold{\Psi}_n + \mathbb{V}\bold{\Psi}_n = E_n \mathbb{M}\bold{\Psi}_n \enspace ,
\end{equation}
or
\[ \hat{\mathbb{K}}\bold{\Psi}_n = E_n \mathbb{M}\bold{\Psi}_n \enspace , \]
where $E_n$ is the $n$th eigenvalue of the system and $\bold{\Psi}_n$ the $n$th eigenvector, and $\hat{\mathbb{K}}$ is an equivalent \emph{stiffness} matrix.

\section{Matrix computation}
The stiffness and mass matrices, for a single element, could be computed as
$$\mathbb{K}_{ij}^{(el)}= \frac{\hslash}{2m} \int\limits_{\Omega_{el}} \nabla N_i \nabla N_j d\Omega_{el}$$
$$\mathbb{M}_{ij}^{(el)}=\int\limits_{\Omega_{el}} N_i N_j d\Omega_{el}$$

\subsection{Potential constant over the element}
With constant $V(x)$ over each element, the potential matrix is

$$\mathbb{V}_{ij}^{(el)}=V(x_m)\mathbb{M}_{ij}^{(el)}, \quad \text{with }x_m\text{ the centroid of each element} \enspace .$$

\subsection{Potential interpolated over the element}
In this case the potential energy operator is 
$$\mathbb{V}_{ij}^{(el)} = \int\limits_{\Omega_{el}}V(x_i)N_i^2N_j d\Omega_{el}$$

\subsection{1D - Linear elements}
The functions $N(r)$ are
\[ N =
 \left(
   \begin{array}{c}
      1-r \\
      r
  \end{array}
\right )\]

so

\[\mathbb{K}_{ij}^{(el)} = \frac{\hslash}{2m}  \begin{pmatrix}
  1 & -1  \\
  -1 & 1 
 \end{pmatrix}, \quad \mathbb{M}_{ij}^{(el)} = \frac{1}{6}  \begin{pmatrix}
  2 & 1 \\
  1 & 2 \end{pmatrix} \enspace ,\]
 
and 
 \[\mathbb{V}_{\text{constant}}^{(el)} = V(x_m)\mathbb{M}^{(el)} \enspace .\]
Making ${V}^{(el)} = (V_1,V_2)$, the value for the potential in each node, we get
\[\mathbb{V}_{\text{interpolated}}^{(el)} = \frac{1}{12}  \begin{pmatrix}
  3V_1 & V1 \\
  V_2 & 3V_2 
 \end{pmatrix} \enspace .\]

\subsubsection{Global Matrices}

(This is not clear in my mind)

The global \emph{equivalent stiffness} matrix is computed as
\begin{align*}
&\mathbb{K}_{ii}=\left(\frac{1}{L_{i-1}} + \frac{1}{L_{i}}\right)\frac{\hslash}{2m} + \frac{1}{3}[V_{i-1} L_{i-1} + V_{i}L_{i}] \qquad \forall\ i\neq 1, i\neq n \enspace, \\
&\mathbb{K}_{i i+1}=\left[ -\frac{1}{L_{i-1}}\frac{\hslash}{2m} + L_{i-1}\frac{V_{i-1}}{6} \right]=
\mathbb{K}_{i+1 i} \qquad \forall\ i\neq n ,\ \text{and}\\
&\mathbb{K}_{11}=\left[ \frac{1}{L_{1}}\frac{\hslash}{2m} + L_{1}\frac{V_{1}}{3} \right] \enspace ,\\
&\mathbb{K}_{nn}=\left[ \frac{1}{L_{n-1}}\frac{\hslash}{2m} + L_{n-1}\frac{V_{n-1}}{3} \right] \enspace ,
\end{align*}
being $L_i$ the length of the $i$th element, if the mesh has a constant size so the previous expressions are simpler.

The global mass matrix is computed as
\begin{align*}
&\mathbb{M}_{ii}=\frac{L_i+L_{i+1}}{3} \qquad \forall\ i\neq 1, i\neq n \enspace, \\
&\mathbb{M}_{i i+1}=\frac{L_i}{6}=\mathbb{M}_{i+1 i} \qquad \forall\ i\neq n ,\ \text{and}\\
&\mathbb{M}_{11}=\frac{L_1}{3} \enspace ,\\
&\mathbb{K}_{nn}=\frac{L_{n-1}}{3} \enspace ,
\end{align*}

\subsection{2D - Linear elements}
The functions $N(r,s)$ are
\[ N =
   \!\left(
    \begin{array}{c}
      1-r-s \\
      r\\
      s
    \end{array}
    \right )\]

so
\[\mathbb{K}_{ij}^{(el)} = \frac{\hslash}{4m}  \begin{pmatrix}
  2 & -1 & -1 \\
  -1 & 1 & 0 \\
  -1 & 0 & 1  
 \end{pmatrix}, \quad \mathbb{M}_{ij}^{(el)} = \frac{1}{24}  \begin{pmatrix}
  2 & 1 & 1 \\
  1 & 2 & 1 \\
  1 & 1 & 2  
 \end{pmatrix} \enspace ,\]
 
and 
 \[\mathbb{V}_{\text{constant}}^{(el)} = V(x_m) \mathbb{M}^{(el)} \enspace .\]
Making ${V}^{(el)} = (V_1,V_2,V_3)$, the value for the potential in each node, we get
\[\mathbb{V}_{\text{interpolated}}^{(el)} = \frac{1}{60}  \begin{pmatrix}
  3V_1 & V_1 & V_1 \\
  V_2  & 3V_2 & V_2 \\
  V_3  & V_3 & 3V_3 
 \end{pmatrix} \enspace .\]
 
\subsection{3D - Linear elements} 
The functions $N(r,s,t)$ are
\[ N =
 \left(
   \begin{array}{c}
      1-r-s-t \\
      r\\
      s\\
      t
  \end{array}
\right )\]
so
\[\mathbb{K}_{ij}^{(el)} = \frac{\hslash}{6m}  \begin{pmatrix}
  3 & -1 & -1 & -1 \\
  -1 & 1 & 0 & 0 \\
  -1 & 0 & 1 & 0  \\
  -1 & 0 & 0 & 1
 \end{pmatrix}, \quad \mathbb{M}_{ij}^{(el)} = \frac{1}{60}  \begin{pmatrix}
  2 & -1 & 0 & -1 \\
  -1 & 4 & 3/2 & 3 \\
  0 & 3/2 & 2 & 3/2 \\
  -1 & 3 & 3/2 & 4 
 \end{pmatrix} \enspace ,\] 
 
and 
 \[\mathbb{V}_{\text{constant}}^{(el)} = V(x_m) \mathbb{M}^{(el)} \enspace .\]
Making ${V}^{(el)} = (V_1,V_2,V_3,V4)$, the value for the potential in each node, we get
\[\mathbb{V}_{\text{interpolated}}^{(el)} = \frac{1}{360}  \begin{pmatrix}
   0    & 5V_1  & 2V_1 & 5V_1 \\
  -5V_2 & 15V_2 & 4V_2 & 10V_2 \\
   0    & 3V_3  & 6V_3 & 3V_3  \\
  -5V_4 & 10V_4 & 4V_4 & 15V_4 
 \end{pmatrix} \enspace . \]
 

%%%%%%
%\begin{thebibliography}{1}
%\bibitem{lsf} Weisstein, Eric W. "Least Squares Fitting." From MathWorld--A Wolfram Web Resource.
%\end{thebibliography}

\end{document}