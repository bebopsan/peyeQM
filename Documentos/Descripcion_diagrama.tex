\documentclass[12pt,letterpaper]{article}
% idioma
\usepackage[utf8x]{inputenc}
\usepackage{ucs}
\usepackage[spanish]{babel}

\usepackage{amsmath}
\usepackage{amsfonts}
\usepackage{amssymb}
%espaciado
\usepackage{setspace}
\onehalfspacing
\setlength{\parindent}{0pt}
\setlength{\parskip}{2.0ex plus0.5ex minus0.2ex}
\usepackage{url} 

\title{Descripción diagrama de FEM}
\author{Santiago Echeverri Chac\'on}
\begin{document} 
\maketitle

\clearpage

\begin{abstract}
Se pretende hacer una descripci\'on del procedimiento que se plantea en el diagrama sobre la soluci\'on de problemas 
de valores propios con  el m\'etodo elementos finitos. 
\end{abstract}

\section{Introducci\'on}

  La particularidad del procedimiento planteado es que parte de una programaci\'on modular. Esto significa que el algoritmo a desarrollar
  lo separaremos en bloques o m\'odulos que ser\'an independientes pero compatibles entre s\'i. Estos bloques podr\'an ser llamados por
  separado desde un programa o script principal que har\'a las veces de usuario. 

Cada m\'odulo representa una de las etapas tradicionales del m\'etodo de elementos finitos:
\begin{itemize}
  \item Preprocesamiento
  \item Procesamiento
  \item Postprocesamiento
\end{itemize}

  Con esto se espera lograr un conjunto de c\'odigos que son m\'as vers\'atiles, f\'aciles de depurar, ampliar y optimizar. 
 
\section{Preprocesamiento}
 La primera etapa llamada ``\textit{preprocesamiento}'', es la etapa en la cual definimos nuestro sistema y le asignamos las condiciones que
 lo rigen.
 Para nuestro caso esto consiste en:

\begin{enumerate}

 \item Definir el dominio espacial del sistema y discretizarlo de acuerdo a la precisi\'on que se dese. El resulado de la
	discretizaci\'on ser\'a un conjunto de subdominios llamados \textit{elementos} y en adelante se llamar\'a \textit{malla} al
	conjunto de elementos que constituyen al dominio inicial.\\	Para el caso de la ecuaci\'on de Scr\"odinger en 1D, definir el
	dominio significa asumir un segmento de recta con una longitud	dada, y discretizarlo implicar\'a seccionar esa	recta en una
	cantidad dada de subdivisiones.\\
	En el problema 2D la longitud dada puede hacer referencia al lado de un pozo cuadrado, o al radio de un pozo circular, y el
	n\'umero de divisiones del dominio es la cantidad de segmentos de \'area que representan el problema 
 \item Establecer las condiciones del problema que est\'an asociadas al dominio del problema, entre ellas:
	\begin{itemize}
	\item Potencial
	\item Fuentes o sumideros
	\end{itemize}
	Para el caso de la ecuaci\'on de Scr\"odinger en 1D, las condiciones implementadas ser\'an del tipo potencial. Esto se traduce en
	asignar un valor de potencial a cada punto o elemento de la malla dada una funci\'on continua o a tramos que describe la naturaleza 
	del fen\'omeno. Por ejemplo un pozo infinito, un oscilador arm\'onico o un pozo del tipo Kronig Penney.   
\end{enumerate}

Una vez realizados estos dos pasos tendremos la informaci\'on fundamental del problema en una representaci\'on discreta que es compatible
con el m\'etodo de soluci\'on implementado en la etapa de procesamiento. Esto, a nivel de programaci\'on significar\'a que la salida de la
etapa de preprocesasmiento son dos matrices que contienen la malla y el potencial evaluado en los elementos de la malla.\\

Estas matrices deber\'an poderse escribir en archivos con un formato predeterminado que sea compatible con archivos procedentes de alg\'un
modulo de preprocesamiento externo, por ejemplo, \textbf{Gmsh}.  
 
\section{Procesamiento}

Siguiendo al preprocesamiento est\'a la etapa de procesamiento, que es la etapa donde se aplica el m\'etodo de soluci\'on dados, el
dominio, sus propiedades y una ecuaci\'on que define el tipo de problema.

Dado que las etapas del procedimiento deben sucederse, es claro que algunos de los argumentos de entrada de la etapa de
procesamiento sean los argumentos de salida de la etapa de preprocesamiento, espec\'ificamente las matrices que definen el dominio y sus
propiedades.  Y teniendo en cuenta que para el m\'etodo de elementos finitos la construcci\'on de las matrices globales y su soluci\'on
dependen de la ecuaci\'on  que se est\'e tratando y de las condiciones de frontera, al modulo de procesamiento se le debe ingresar un tercer
argumento que permita seleccionar entre distintas operaciones.

Para el caso de la ecuaci\'on de Scr\"odinger se plantean 3 algoritmos distintos que hacen referencia la soluci\'on por FEM
asumiendo condiciones de frontera que permiten describir un problema en su forma local o bajo dos condiciones de periodicidad distintas. Con
ello adquirimos una herramienta que permite describir situaciones cu\'anticas en materiales cristalinos, y compararlas con sus equivalentes
locales o no peri\'odicos. 
Por otra parte, y sabiendo que la ecuaci\'on de Schr\"odinger se traduce en un problema de valores y vectores propios con una
superposici\'on de infinitas soluciones, se plantea el argumento de entrada nVals, como indicador de la cantidad de soluciones a evaluar.

Finalmente, en la etapa de procesamiento se 




\end{document}
